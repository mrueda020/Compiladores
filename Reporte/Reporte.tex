\documentclass[12pt,twoside]{article}
\usepackage{amsmath, amssymb}
\usepackage{amsmath}
\usepackage[active]{srcltx}
\usepackage{amssymb}
\usepackage{amscd}
\usepackage{makeidx}
\usepackage{amsthm}
\usepackage{algpseudocode}
\usepackage{algorithm}
\usepackage{graphicx}
\usepackage{mathtools}
\usepackage{skmath}
\usepackage{multicol}
\usepackage{subfigure}
\usepackage[utf8]{inputenc}
\usepackage[spanish]{babel}
\renewcommand{\baselinestretch}{1}
\setcounter{page}{1}
\setlength{\textheight}{21.6cm}
\setlength{\textwidth}{14cm}
\setlength{\oddsidemargin}{1cm}
\setlength{\evensidemargin}{1cm}
\pagestyle{myheadings}
\thispagestyle{empty}
\date{}
\begin{document}
\centerline{\bf Compiladores, Sem: 2020-2, 3CV8, Proyecto}
\centerline{}
\centerline{}
\begin{center}
\Large{\textsc{Proyecto: Compilador básico para el lenguaje C.}}
\end{center}
\centerline{}
\centerline{\bf {Rueda Carbajal Miguel}}
\centerline{}
\centerline{Escuela Superior de C\'omputo}
\centerline{Instituto Polit\'ecnico Nacional, M\'exico}
\centerline{$mrueda020@hotmail.com$}
\newtheorem{Theorem}{\quad Theorem}[section]
\newtheorem{Definition}[Theorem]{\quad Definition}
\newtheorem{Corollary}[Theorem]{\quad Corollary}
\newtheorem{Lemma}[Theorem]{\quad Lemma}
\newtheorem{Example}[Theorem]{\quad Example}
\bigskip
\textbf{Resumen: }\\
En este reporte se explicaran las partes que componen a un compilador, ademas se explicara como se fue implementando cada una de esas etapas.  
\textbf{}\\
\textbf{}\\
\textbf{Palabras Clave: }
Compilador, Traductor, Expresiones regulares 
\section{Introducción}
\text{}\\
Un compilador es un programa informático, que se encarga de traducir (compilar) el código fuente de cualquier aplicación que se esté desarrollando. En pocas palabras, es un software que se encarga de traducir el programa hecho en lenguaje de programación, a un lenguaje de máquina que pueda ser comprendido por el equipo y pueda ser procesado o ejecutado por este.

\section{Conceptos B\'asicos}
\text{}\\





\text{}\\
\clearpage

\clearpage
\section{Conclusiones}







\end{document}


